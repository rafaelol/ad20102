\documentclass[a4paper,10pt]{article}
\usepackage{graphicx}

\usepackage[utf8]{inputenc}
\usepackage[portuguese,brazil]{babel}
\usepackage[T1]{fontenc}

\usepackage{indentfirst}
\usepackage{url}


\begin{document}


\begin{titlepage}

\begin{minipage}{0.2\linewidth}
 \includegraphics[]{./minerva.png}
\end{minipage}
\begin{minipage}{0.8\linewidth}
 \textbf{Universidade Federal do Rio de Janeiro}\\
 Instituto de Matemática\\
 Departamento de Ciência da Computação\\
 \rule{0.8\linewidth}{0.5mm}\\
 Rio de Janeiro, RJ - Brasil
\end{minipage}

\begin{center}

\vspace{2cm}

\Large
Trabalho de Simulaçao: Implementação e análise de um simulador.

\vspace{1cm}

\large

\_\_\_\_\_\_\_\_\_\_\_\_\_\_\_\_\_\_\_\_\_\_\_\_\_\_\_\_\_\_\_\_\_\_\_\_\_\_\_\\
Bruno C. Buss\\(Implementação e documentação do código)\\

\vspace{0.5cm}

\_\_\_\_\_\_\_\_\_\_\_\_\_\_\_\_\_\_\_\_\_\_\_\_\_\_\_\_\_\_\_\_\_\_\_\_\_\_\_\\
Felipe P. Martinez\\()\\

\vspace{0.5cm}

\_\_\_\_\_\_\_\_\_\_\_\_\_\_\_\_\_\_\_\_\_\_\_\_\_\_\_\_\_\_\_\_\_\_\_\_\_\_\_\\
Rafael O. Lopes\\(Implementação, documentação do código e relatório final)\\

\vspace{0.5cm}

\_\_\_\_\_\_\_\_\_\_\_\_\_\_\_\_\_\_\_\_\_\_\_\_\_\_\_\_\_\_\_\_\_\_\_\_\_\_\_\\
Yanko G. Oliveira\\()\\

\vspace{0.5cm}

\vspace{1cm}

Relatório gerado em \today

\normalsize
\end{center}

\vfill

\begin{flushright}
Disciplina: Avaliação e Desempenho 2010/2\\
Professor: Paulo Henrique de Aguiar Rodrigues\\
\end{flushright}

\vspace{2cm}

\end{titlepage}

\pagebreak

\tableofcontents
\pagebreak

\section{Introdução}
\subsection{Funcionamento Geral do Simulador}
    O simulador funciona de maneira simples. Ao executar o programa, parâmetros são passados para que se inicie a simulação. Tais parâmetros são:

\begin {itemize}
\item Modo: O usuário escolhe qual o parâmetro de execução da simulação, se será replicativo ou batch.

\item Quantidade de rodadas: O usuário decide a quantidade de rodadas. A quantidade de rodadas deve ser superior a 10, pois apenas com quantidades de rodadas
superiores a 10 temos um valor assintótico na tabela t-student para o cálculo do intervalo de confiança.

\item Tamanho da rodada: Determina a quantidade de fregueses típicos em cada rodada.

\item Tamanho da fase transiente: Determina a quantidade de fregueses típicos existirão na fase transiente, de forma a termos um sistema em equilíbrio nas rodadas.

\item Tipo da fila 1: O tipo da fila pode ser FCFS(fila), ou LCFS(pilha).

\item Tipo da fila 2: Assim como a fila 1, a fila 2 pode ser FCFS ou LCFS.

\item Taxa lambda: Determina a taxa de chegada de chegada de fregueses no sistema.

\item Taxa mi: Determina a taxa de serviço de fregueses no sistema.
\end {itemize}

    Além desses parâmetros, existem outros dois parâmetros opcionais para o simulador, que determinam a semente geradora de chegadas e semente geradora
de tempo de serviço.

    Após a passagem de parâmetros, o simulador se inicia. A cada rodada, o simulador fica em loop aguardando eventos acontecerem. O loop se encerra quando for servido pelo
servidor a quantidade determinada pelo parâmetro como tamanho da rodada. Os eventos que acontecem são:

\begin {itemize}
\item Chegada de um freguês na fila 1: Um freguês chega e é inserido na fila1, de acordo com a disciplina de atendimento da fila. Se for FCFS, é inserido no final da fila
se for LCFS, adicionado no início da fila.

\item Término do serviço do freguês no servidor: No momento que um freguês termina seu atendimento no servidor, é verificado qual a fila original dele. Se for a fila 1,
ele é adicionado na fila 2, de acordo com a disciplina de atendimento. Se o freguês era da fila 2, dados estatísticos são coletados e é retirado do sistema.
\end {itemize}

    Após a verificação de qual evento ocorreu, verifica-se se o servidor está vazio. Caso esteja vazio, é inserido no servidor um freguês, de acordo com as regras
determinadas: sempre que houver um freguês na fila 1, ele é atendido, e os fregueses da fila 2 só serão atendidos no momento em que não houver ninguém na fila 1.

    Após a execução de todas as rodadas, dados estatísticos são coletados. No final da execução de todas as rodadas, é calculado o intervalo de confiança para cada
variável aleatória. O programa se encerra com a disponibilização dos parâmetros, incluindo as sementes geradoras, para o caso de repetir a simulação, caso desejado.

\subsection{Linguagem de Programação Utilizada}
    Foi utilizada a linguagem de programação C++. O compilador utilizado para desenvolver o sistema foi o GCC 4.4, presente em qualquer distribuição atualizada do Linux, de forma que o sistema pode ser compilado em qualquer ambiente que possua o GCC 4.4 ou superior instalado.

    A escolha da linguagem C++ se deve à possibilidade de lidar com orientação a objetos, tornando o código mais sucinto e de melhor entendimento. Um outro fato é que C++ é a linguagem com que os membros do grupo estão melhor familiarizados, tornando o desenvolvimento do sistema mais fácil.

    O sistema não tem garantias de funcionamento no sistema operacional Windows, porque os compiladores de C++ para Windows(MinGW) utilizam versões anteriores do GCC, que não possuem recursos utilizados para o desenvolvimento do sistema. Um dos recursos é o responsável pela geração de sementes pseudo-aleatórias.
\subsection{Estruturas Internas Utilizadas}
\subsection{Implementação da Lista de Eventos}
\subsection{Geração das Variáveis Aleatórias}
\subsection{Método de análise de resultados}
\subsection{Parâmetros Utilizados}
\subsection{Tempo Gasto Para Simulação}

\pagebreak

\section{Teste de Correção}

\pagebreak

\section{Estimativa da fase transiente}

\pagebreak

\section{Tabelas com resultados e comentários pertinentes}

\pagebreak

\section{Otimização}

\pagebreak

\section{Conclusão}

\pagebreak

\end{document}
