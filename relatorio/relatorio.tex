\documentclass[a4paper,10pt]{article}
\usepackage{graphicx}

\usepackage[utf8]{inputenc}
\usepackage[portuguese,brazil]{babel}
\usepackage[T1]{fontenc}

\usepackage{indentfirst}
\usepackage{url}


\begin{document}


\begin{titlepage}

\begin{minipage}{0.2\linewidth}
 \includegraphics[]{./minerva.png}
\end{minipage}
\begin{minipage}{0.8\linewidth}
 \textbf{Universidade Federal do Rio de Janeiro}\\
 Instituto de Matemática\\
 Departamento de Ciência da Computação\\
 \rule{0.8\linewidth}{0.5mm}\\
 Rio de Janeiro, RJ - Brasil
\end{minipage}

\begin{center}

\vspace{2cm}

\Large
Trabalho de Simulaçao: Implementação e análise de um simulador.

\vspace{1cm}

\large

\_\_\_\_\_\_\_\_\_\_\_\_\_\_\_\_\_\_\_\_\_\_\_\_\_\_\_\_\_\_\_\_\_\_\_\_\_\_\_\\
Bruno C. Buss\\(Implementação, documentação do código e relatório final)\\

\vspace{0.5cm}

\_\_\_\_\_\_\_\_\_\_\_\_\_\_\_\_\_\_\_\_\_\_\_\_\_\_\_\_\_\_\_\_\_\_\_\_\_\_\_\\
Felipe P. Martinez\\(Implementação, documentação do código e relatório final)\\

\vspace{0.5cm}

\_\_\_\_\_\_\_\_\_\_\_\_\_\_\_\_\_\_\_\_\_\_\_\_\_\_\_\_\_\_\_\_\_\_\_\_\_\_\_\\
Rafael O. Lopes\\(Implementação, documentação do código e relatório final)\\

\vspace{0.5cm}

\_\_\_\_\_\_\_\_\_\_\_\_\_\_\_\_\_\_\_\_\_\_\_\_\_\_\_\_\_\_\_\_\_\_\_\_\_\_\_\\
Yanko G. Oliveira\\(Implementação, documentação do código e relatório final)\\

\vspace{0.5cm}

\vspace{1cm}

Relatório gerado em \today

\normalsize
\end{center}

\vfill

\begin{flushright}
Disciplina: Avaliação e Desempenho 2010/2\\
Professor: Paulo Henrique de Aguiar Rodrigues\\
\end{flushright}

\vspace{2cm}

\end{titlepage}

\pagebreak

\tableofcontents
\pagebreak

\section{Introdução}
\subsection{Funcionamento Geral do Simulador}
    O simulador é executado por linha de comando (comando padrão: "./cmulator"). Este aguarda diversos parâmetros que podem ser passados pela linha de comando ou, caso isso não seja feito, são pedidos em tempo de execução. Tais parâmetros são:

\begin {itemize}
\item \textbf{Modo:} O usuário escolhe qual o parâmetro de execução da simulação, se será replicativo ou batch. Na linha de comando: \emph{--modo} ou \emph{-m}. Opções possíveis: \emph{"Batch"} ou \emph{"Replicativo"};

\item \textbf{Quantidade de rodadas:} O usuário decide a quantidade de rodadas; esta deve ser superior a 10, pois apenas assim temos um valor assintótico na tabela t-student para o cálculo do intervalo de confiança. Na linha de comando: \emph{--n\_rodadas} ou \emph{-n}. Opções possíveis: números inteiros;

\item \textbf{Tamanho da rodada:} Determina a quantidade de fregueses típicos em cada rodada. Na linha de comando: \emph{--t\_rodada} ou \emph{-r}. Opções possíveis: números inteiros;

\item \textbf{Tamanho da fase transiente:} Determina a quantidade de fregueses típicos a serem considerados pertencentes à fase transiente, de forma a considerarmos apenas dados coletados após o sistema estar em equilíbrio. Na linha de comando: \emph{--t\_transiente} ou \emph{-t}. Opções possíveis: números inteiros;

\item \textbf{Tipo da fila 1:} O tipo da fila pode ser FCFS(fila), ou LCFS(pilha). Na linha de comando: \emph{--fila\_1} ou \emph{-1}. Opções possíveis: \emph{"FCFS"} ou \emph{"LCFS"};

\item \textbf{Tipo da fila 2:} Assim como a fila 1, a fila 2 pode ser FCFS ou LCFS. Na linha de comando: \emph{--fila\_2} ou \emph{-2}. Opções possíveis: \emph{"FCFS"} ou \emph{"LCFS"};

\item \textbf{Taxa $\lambda$:} Determina a taxa de chegada de chegada de fregueses no sistema. Na linha de comando: \emph{--tx\_lambda} ou \emph{-l}. Opções possíveis: números reais;

\item \textbf{Taxa $\mu$:} Determina a taxa de serviço de fregueses no sistema. Na linha de comando: \emph{--tx\_mi} ou \emph{-m}. Opções possíveis: números reais.
\end {itemize}

    Além desses parâmetros, que são obrigatoriamente pedidos, existem outros parâmetros opcionais para o simulador, a serem passados pela linha de comando:

\begin {itemize}
\item \textbf{Semente para gerador de chegadas:} semente utilizada para o gerador de números pseudo-aleatórios responsável por gerar os tempos de chegada. Na linha de comando: \emph{--seed\_gerador\_chegadas} ou \emph{-c}. Opções possíveis: números reais;

\item \textbf{Semente para gerador de tempos de serviço:} semente utilizada para o gerador de números pseudo-aleatórios responsável por gerar os tempos de serviço. Na linha de comando: \emph{--seed\_gerador\_tempo\_servico} ou \emph{-x}. Opções possíveis: números reais;

\item \textbf{Sobre:} imprime informações sobre o simulador e seus autores. Na linha de comando: \emph{--sobre} ou \emph{-s}. Não recebe parâmetros;

\item \textbf{Ajuda:} imprime todos os comandos possíveis e suas descrições. Na linha de comando: \emph{--ajuda} ou \emph{-a}. Não recebe parâmetros;

\item \textbf{Modo verborrágico:} roda o simulador imprimindo diferentes níveis de detalhe sobre as operações sendo executadas internamente. Na linha de comando: \emph{--verbose} ou \emph{-v} Opções: 0,1 ou 2;
\end {itemize}

    Após a entrada de parâmetros, o simulador se inicia. Este utiliza eventos discretos, mas tempo contínuo, logo, os eventos acontecem em instantes de tempo com valores numéricos reais. A cada rodada, o simulador fica em loop aguardando eventos acontecerem. O loop se encerra quando a quantidade de fregueses determinada pelo parâmetro \emph{"tamanho da rodada"} for servida. Os eventos que acontecem são:

\begin {itemize}
\item Chegada de um freguês na fila 1: Um freguês chega e é inserido na fila1, de acordo com a disciplina de atendimento da fila. Se for FCFS, é inserido no final da fila
se for LCFS, adicionado no início da fila.

\item Término do serviço do freguês no servidor: No momento que um freguês termina seu atendimento no servidor, é verificado qual a fila original dele. Se for a fila 1,
ele é adicionado na fila 2, de acordo com a disciplina de atendimento. Se o freguês era da fila 2, dados estatísticos são coletados e é retirado do sistema.
\end {itemize}

    Após a verificação de qual evento ocorreu, verifica-se se o servidor está vazio. Caso esteja vazio, é inserido no servidor um freguês, de acordo com as regras
determinadas: sempre que houver um freguês na fila 1, ele é atendido, e os fregueses da fila 2 só serão atendidos no momento em que não houver ninguém na fila 1.

    Após a execução de todas as rodadas, dados estatísticos são coletados. No final da execução de todas as rodadas, é calculado o intervalo de confiança para cada
variável aleatória. O programa se encerra com a disponibilização dos parâmetros, incluindo as sementes geradoras, para o caso de repetir a simulação, caso desejado.

\subsection{Linguagem de Programação Utilizada}
    Foi utilizada a linguagem de programação C++. O compilador utilizado para desenvolver o sistema foi o GCC 4.4, presente em qualquer distribuição atualizada do Linux, de forma que o sistema pode ser compilado em qualquer ambiente que possua o GCC 4.4 ou superior instalado.

    A escolha da linguagem C++ se deve à possibilidade de lidar com orientação a objetos, tornando o código mais sucinto e de melhor entendimento. Um outro fato é que C++ é a linguagem com que os membros do grupo estão melhor familiarizados, tornando o desenvolvimento do sistema mais fácil.

    O sistema não tem garantias de funcionamento no sistema operacional Windows, porque os compiladores de C++ para Windows(MinGW) utilizam versões anteriores do GCC, que não possuem recursos utilizados para o desenvolvimento do sistema. Um dos recursos é o responsável pela geração de sementes pseudo-aleatórias.
\subsection{Estruturas Internas Utilizadas}
% Aqui comentamos sobre as classes do projeto, e alguns detalhes, como a utilização do rand do linux, ou de
\subsection{Implementação da Lista de Eventos}
% Creio que essa seção pode morrer, pois a lista de eventos já foi mencionada antes. Talvez possamos desenvolver melhor a parte de eventos lá.
\subsection{Geração das Variáveis Aleatórias}
% Aqui dizemos as variáveis aleatórias que temos, e como elas são calculadas(mencionar a inversa da exponencial etc)
\subsection{Coleta de Estatísticas}
% Falamos sobre a coleta de estatísticas e cálculo delas(Variância e IC).
\subsection{Método de análise}
% Aqui mencionamos que decidimos testar os dois métodos: Batch e Replicativo. E também falar um pouco sobre eles.
% Acredito que essa seção pode morrer.
%\subsection{Parâmetros Utilizados}
% Essa sub-seção eu peguei do relatório do Jonas. No projeto deles, eles determinaram um valor para alguns parâmetros que nós tomamos como fixos. Creio que isso pode morrer.
\subsection{Tempo Gasto Para Simulação}
% Tempo que gastamos nos testes

\pagebreak

\section{Teste de Correção}
% Aqui temos que provar que está correto.
% Para fazer isso, devemos comparar os resultados da execução determinística com os resultados da execução exponencial.
\pagebreak

\section{Estimativa da fase transiente}
% Benchmark do Buss
\pagebreak

\section{Tabelas com resultados e comentários pertinentes}
% Tabelas, tabelas e tabelas. =)
\pagebreak

\section{Otimização}
% O que falar exatamente aqui?
\pagebreak

\section{Conclusão}

\pagebreak

\end{document}
