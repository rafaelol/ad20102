\documentclass[a4paper,10pt]{article}
\usepackage{graphicx}

\usepackage[utf8]{inputenc}
\usepackage[portuguese,brazil]{babel}
\usepackage[T1]{fontenc}

\usepackage{indentfirst}
\usepackage{url}


\begin{document}


\begin{titlepage}

\begin{minipage}{0.2\linewidth}
 \includegraphics[]{./minerva.png}
\end{minipage}
\begin{minipage}{0.8\linewidth}
 \textbf{Universidade Federal do Rio de Janeiro}\\
 Instituto de Matemática\\
 Departamento de Ciência da Computação\\
 \rule{0.8\linewidth}{0.5mm}\\
 Rio de Janeiro, RJ - Brasil
\end{minipage}

\begin{center}

\vspace{2cm}

\Large
Trabalho de Simulaçao: Implementação e análise de um simulador.

\vspace{1cm}

\large

\_\_\_\_\_\_\_\_\_\_\_\_\_\_\_\_\_\_\_\_\_\_\_\_\_\_\_\_\_\_\_\_\_\_\_\_\_\_\_\\
Bruno C. Buss\\(Implementação e documentação)\\

\vspace{0.5cm}

\_\_\_\_\_\_\_\_\_\_\_\_\_\_\_\_\_\_\_\_\_\_\_\_\_\_\_\_\_\_\_\_\_\_\_\_\_\_\_\\
Felipe P. Martinez\\()\\

\vspace{0.5cm}

\_\_\_\_\_\_\_\_\_\_\_\_\_\_\_\_\_\_\_\_\_\_\_\_\_\_\_\_\_\_\_\_\_\_\_\_\_\_\_\\
Rafael O. Lopes\\(Implementação e documentação)\\

\vspace{0.5cm}

\_\_\_\_\_\_\_\_\_\_\_\_\_\_\_\_\_\_\_\_\_\_\_\_\_\_\_\_\_\_\_\_\_\_\_\_\_\_\_\\
Yanko G. Oliveira\\()\\

\vspace{0.5cm}

\vspace{1cm}

Relatório gerado em \today

\normalsize
\end{center}

\vfill

\begin{flushright}
Disciplina: Avaliação e Desempenho 2010/2\\
Professor: Paulo Henrique de Aguiar Rodrigues\\
\end{flushright}

\vspace{2cm}

\end{titlepage}

\pagebreak

\tableofcontents
\pagebreak

\section{Introdução}
\subsection{Funcionamento Geral do Simulador}
\subsection{Linguagem de Programação Utilizada}
    Foi utilizada a linguagem de programação C++. O compilador utilizado para desenvolver o sistema foi o GCC 4.4, presente em qualquer distribuição atualizada do Linux, de forma que o sistema pode ser compilado em qualquer ambiente que possua o GCC 4.4 ou superior instalado.
    
    A escolha da linguagem C++ se deve à possibilidade de lidar com orientação a objetos, tornando o código mais sucinto e de melhor entendimento. Um outro fato é que C++ é a linguagem com que os membros do grupo estão melhor familiarizados, tornando o desenvolvimento do sistema mais fácil.
    
    O sistema não tem garantias de funcionamento no sistema operacional Windows, porque os compiladores de C++ para Windows(MinGW) utilizam versões anteriores do GCC, que não possuem recursos utilizados para o desenvolvimento do sistema. Um dos recursos é o responsável pela geração de sementes pseudo-aleatórias.
\subsection{Estruturas Internas Utilizadas}
\subsection{Implementação da Lista de Eventos}
\subsection{Geração das Variáveis Aleatórias}
\subsection{Método de análise de resultados}
\subsection{Parâmetros Utilizados}
\subsection{Tempo Gasto Para Simulação}

\pagebreak

\section{Teste de Correção}

\pagebreak

\section{Estimativa da fase transiente}

\pagebreak

\section{Tabelas com resultados e comentários pertinentes}

\pagebreak

\section{Otimização}

\pagebreak

\section{Conclusão}

\pagebreak

\end{document}
